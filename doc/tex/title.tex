
\title{Entwicklung eines automatischen Timing-Messgeräts für Kameras
\\\large
\hfill \break
Pho21 Bachelorthesis der Fachhochschule Graubünden \\in Zusammenarbeit mit Astrivis Technologies Ltd
}

\author{}


\maketitle

\vspace{-0.5cm}

\begin{figure}[!h]
    \centering
    \includegraphics[height=20em]{../images/phone/gerät-edit.jpg}
\end{figure}

\vspace{0.5cm}

\begin{longtable}[l]{ @{} >{\RaggedRight\hspace{0pt}} p{.25\linewidth}p{.7\linewidth} @{} }
    Autor:
    & Raphael Seitz
    \\& E-Mail: r.seitz2001@gmail.com
    \\ 
    \\ Referent:
    & Prof. Dr. Philipp Roebrock, FH Graubünden,
    \\& Institut für Photonics und Robotics

    \\ Korreferent:
    & Olivier Saurer, Astrivis Technologies Ltd

    \\ Studiengang:
    & BSc Photonics, Institut für Photonics und Robotics, FH Graubünden

    \\ Start der Thesis: & 19.02.2024
    \\ Abgabe der Thesis: & 09.08.2024
    % ENDE
    \addtocounter{table}{-1}\setcounter{enumi}{0}
\end{longtable}

\thispagestyle{empty} % remove page number

\clearpage
\begin{abstract}
    Viele Computervision-Anwendungen sind auf ein korrektes Timing der aufgenommenen Bilder angewiesen.
    Es wurde ein Timing-Messgerät entwickelt, mit welchem die Bildwiederholrate und Belichtung über die Zeit detektiert werden können.
    Dies geschieht über das Analysieren der Änderung vom Wandern lassen einer LED in einer 4x4 Anordnung.
    Das Gerät lässt sich über ein USB-Kabel mit einem Computer verbinden und über die serielle Schnittstelle konfigurieren.
    Während dem Messvideo muss die Frequenz des Messgerätes leicht von der Kamera abweichen und die Position und die Lichtverhältnisse so unveränderlich wie möglich bleiben.
    Die Auswertung geschieht im Nachhinein durch eine Software geschrieben in Python.
    Darin wird automatisch die Position des Gerätes gefunden und die Helligkeiten der LEDs kalibriert, um robuste Resultate zu erhalten.

    Dies funktioniert erst für ein Messystem bestehend aus je einer Kamera mit Global Shutter und einem Messgerät.
    Eine Kommunikation zwischen Messgeräten ist gegeben aber noch nicht implementiert.
    Die Auswertung kann und darf derzeit nur ein Messgerät detektieren.
    Im Fall, wenn zwei Kameras dasselbe Messgerät zur gleichen Zeit filmen, geschieht während der Auswertung kein Austausch der Messwerte zwischen den einzelnen Videos.

\end{abstract}

