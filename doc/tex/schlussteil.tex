% INHALT: {{{
% In den Schlussteil der Arbeit gehören eine Zusammenfassung der wichtigsten Ergebnisse,
% die Interpretation der Resultate und der Vergleich mit eventuell bereits vorhandenen Resultaten.
% Hinzu kommt eine kritische Beurteilung der eigenen Arbeit (Fragestellung und methodisches Vorgehen im Vergleich mit den erreichten Ergebnissen),
% offene Fragen und ein Ausblick (Hinweise auf weiterführende Themenstellungen).
% }}}

\subsection{Zusammefassung der Ergebnisse}

\begin{longtable}[l]{ @{} >{\RaggedRight\hspace{0pt}} lp{.96\linewidth} @{} }
    \textbullet & Die Kodierung der Zeit geschieht über Änderungen in der Position der LEDs (\secref{sec:Zentrum-def-laufenden-Leuchtdiode}).
    \\\textbullet & Die Kodierung der Belichtungszeit ist ablesbar aus der Anzahl an aktiven LEDs (\secref{sec:Breite-der-laufenden-Leuchtdiode}).
    \addtocounter{table}{-1}\setcounter{enumi}{0}
\end{longtable}

\subsection{Interpretation der Resultate}

Die in \appref{app:results} gezeigten Resultate haben diverse Einflüsse:
\begin{longtable}[l]{ @{} >{\RaggedRight\hspace{0pt}} lp{.96\linewidth} @{} }
    \textbullet & Eine Änderung der Belichtungszeit über die Messdauer spielt vermutlich zusammen mit sich ändernen Lichtverhältnissen, denn dies ist sehr anfällig darüber.
    \\\textbullet & Die FPS bleiben in der Regel sehr gleichmässig, bei Ausreissern gingen entweder Bilder verloren (weniger FPS),
    die Frequenzen waren nicht gut abgestimmt (weniger oder mehr FPS), 
    die Kamera verlor sehr viele Bilder (mehr FPS)
    oder die Kamera wiederholte gewisse Bilder (mehr FPS).
    Ein Fall, welcher nicht detektiert werden kann, ist wenn ein Vielfaches von \eqref{eq:Delta-f} Bildern verloren gehen.
    \\\textbullet & Wenn die Frequenzen zwischen Messgerät und Kamera gut eingestellt sind, weicht die ermittelte FPS
    in der Regel weniger als 0.3\% von der vorgegebenen ab. Die grösste Abweichung ist sichtbar in den Resultaten \hyperlinkXY{hyp:k-aos-285}.
    \\\textbullet & Die Kalibration verbessert in der Regel die Belichtungszeit zu den Winkeln, was die Effektivität dessen darstellt.
    Je regelmässiger dies ist, desto Wahrheitsgetreuer sind alle Resultate bei einer stabilen Kamera.
    \addtocounter{table}{-1}\setcounter{enumi}{0}
\end{longtable}

\subsection{Kritische Beurteilung}

Verschiedene Punkte, welche mit dem aktuellen Messgerät und der aktuellen Auswertungssoftware nicht funktionieren, sind folgende:

\begin{longtable}[l]{ @{} >{\RaggedRight\hspace{0pt}} lp{.96\linewidth} @{} }
% - kein Rolling shutter möglich
    \textbullet & Bei Videos mit Rolling Shutter liefert die Auswertesoftware, wenn überhaupt das Orientierungsmuster erkannt wird, falsche Resultate.
% - keine 100% belichtungszeit möglich
    \\\textbullet & Bei 100\% Belichtungszeit von Bildern kann keine Position im \hyperlinkXY{hyp:mode-rider} Modus entdeckt werden.
    Dies kann zum Beispiel bei einer Hochgeschwindigkeitskamera der Fall sein.
% - falls 1/FPS (bzw die Hälfte davon) Bilder verloren gehen -> "kein" fehler...
    \\\textbullet & Es ist wichtig, dass die Frequenz des Messgerätes korrekt eingestellt ist (\secref{sec:Gut-Eingestellt}).
% - CAN-Bus noch nicht
    \\\textbullet & Kommunikation zwischen Messgeräten ist (noch) nicht implementiert.
% - Speichern der Einstellungen bei shutdown im gerät?
    \\\textbullet & Vorgenommene Konfigurationen gehen beim Reset oder Hochfahren verloren.
% - keine variable FPS / belichtungszeiten -> anfällig gegen sich ändernde Belichtungszeiten beim Aufnehmen
    \\\textbullet & Um die Korrektheit der Auswertung zu erhöhen ist es sehr wichtig, dass sich die Lichtverhältnisse über die Zeit nicht ändern.
    % ENDE
    \addtocounter{table}{-1}\setcounter{enumi}{0}
\end{longtable}

\subsection{Offene Fragen}

Eine Auswertung eines Mehrkamerasystems oder Präsenz von mehrfachen Messgeräte je Video konnte aus Zeitgründen nicht näher betrachtet werden.

Ansätze um LEDs im Raum zu tracken \cite{6044360} konnten aus Zeitgründen nicht angeschaut werden.
%denn das zu entwickelnde Gerät und die zu untersuchende Kamera bleiben statisch im Raum.

\subsection{Ausblick}

Es wurde erwähnt, dass die Auswertung nicht für Kameras mit Rolling Shutter funktioniert.
Würde die Orientierung des Messgerätes zur Kamera respektiert und die Auswertungssoftware darauf abgestimmt und erweitert werden,
könnte so eventuell trotzdem eine Auswertung vonstatten gehen.

Würden zwei oder mehr Kameras ein Messgerät zur gleichen Zeit filmen, könnte mit einer erweiterten Auswertungssoftware
die Stabilität der Kameras oder der Auswertungssoftware näher beurteilt werden.

Bei zwei oder mehr im Bild aufgenommenen Messgeräten,
deren Frequenzen leicht versetzt eingestellt sind, könnte
$s_\text{maxskip}$ \eqref{eq:s_maxskip} erhöht werden.

Um die Auswertungssoftware auf Korrektheit zu Überprüfen, könnten anstelle von aufgenommenen, kontrolliert Computer-Generierte Videos erstellt 
und ausgewertet werden.

